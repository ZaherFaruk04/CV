%-------------------------
% Resume in Latex
% Author : Zaher Faruk
% Based off of: https://github.com/sb2nov/resume
% License : MIT
%------------------------

\documentclass[letterpaper,11pt]{article}

\usepackage{latexsym}
\usepackage[empty]{fullpage}
\usepackage{titlesec}
\usepackage{marvosym}
\usepackage[usenames,dvipsnames]{color}
\usepackage{verbatim}
\usepackage{enumitem}
\usepackage[hidelinks]{hyperref}
\usepackage{fancyhdr}
\usepackage[english]{babel}
\usepackage{tabularx}
\input{glyphtounicode}


%----------FONT OPTIONS----------
% sans-serif
% \usepackage[sfdefault]{FiraSans}
% \usepackage[sfdefault]{roboto}
% \usepackage[sfdefault]{noto-sans}
% \usepackage[default]{sourcesanspro}

% serif
% \usepackage{CormorantGaramond}
% \usepackage{charter}


\pagestyle{fancy}
\fancyhf{} % clear all header and footer fields
\fancyfoot{}
\renewcommand{\headrulewidth}{0pt}
\renewcommand{\footrulewidth}{0pt}

% Adjust margins
\addtolength{\oddsidemargin}{-0.5in}
\addtolength{\evensidemargin}{-0.5in}
\addtolength{\textwidth}{1in}
\addtolength{\topmargin}{-.5in}
\addtolength{\textheight}{1.0in}

\urlstyle{same}

\raggedbottom
\raggedright
\setlength{\tabcolsep}{0in}

% Sections formatting
\titleformat{\section}{
  \vspace{-4pt}\scshape\raggedright\large
}{}{0em}{}[\color{black}\titlerule \vspace{-5pt}]

% Ensure that generate pdf is machine readable/ATS parsable
\pdfgentounicode=1

%-------------------------
% Custom commands
\newcommand{\resumeItem}[1]{
  \item\small{
    {#1 \vspace{-2pt}}
  }
}

\newcommand{\resumeSubheading}[4]{
  \vspace{-2pt}\item
    \begin{tabular*}{0.97\textwidth}[t]{l@{\extracolsep{\fill}}r}
      \textbf{#1} & #2 \\
      \textit{\small#3} & \textit{\small #4} \\
    \end{tabular*}\vspace{-7pt}
}

\newcommand{\resumeSubSubheading}[2]{
    \item
    \begin{tabular*}{0.97\textwidth}{l@{\extracolsep{\fill}}r}
      \textit{\small#1} & \textit{\small #2} \\
    \end{tabular*}\vspace{-7pt}
}

\newcommand{\resumeProjectHeading}[2]{
    \item
    \begin{tabular*}{0.97\textwidth}{l@{\extracolsep{\fill}}r}
      \small#1 & #2 \\
    \end{tabular*}\vspace{-7pt}
}

\newcommand{\resumeSubItem}[1]{\resumeItem{#1}\vspace{-4pt}}

\renewcommand\labelitemii{$\vcenter{\hbox{\tiny$\bullet$}}$}

\newcommand{\resumeSubHeadingListStart}{\begin{itemize}[leftmargin=0.15in, label={}]}
\newcommand{\resumeSubHeadingListEnd}{\end{itemize}}
\newcommand{\resumeItemListStart}{\begin{itemize}}
\newcommand{\resumeItemListEnd}{\end{itemize}\vspace{-5pt}}

%-------------------------------------------
%%%%%%  RESUME STARTS HERE  %%%%%%%%%%%%%%%%%%%%%%%%%%%%


\begin{document}

%----------HEADING----------
% \begin{tabular*}{\textwidth}{l@{\extracolsep{\fill}}r}
%   \textbf{\href{http://sourabhbajaj.com/}{\Large Sourabh Bajaj}} & Email : \href{mailto:sourabh@sourabhbajaj.com}{sourabh@sourabhbajaj.com}\\
%   \href{http://sourabhbajaj.com/}{http://www.sourabhbajaj.com} & Mobile : +1-123-456-7890 \\
% \end{tabular*}

\begin{center}
    \textbf{\Huge \scshape Zaher Faruk} \\ \vspace{1pt}
    \small 07553064043 $|$ \href{mailto:zaherfaruk@gmail.com}{\underline{zaherfaruk04@gmail.com}} $|$ 
    \href{https://www.linkedin.com/in/zaher-faruk-b6546231a/}{\underline{linkedin.com/in/zaher-faruk-b6546231a/}} $|$
    \href{https://github.com/ZaherFaruk04}{\underline{github.com/ZaherFaruk04}}
\end{center}


%-----------EDUCATION-----------
\section{Education}
  \resumeSubHeadingListStart
    \resumeSubheading
      {The University of Salford}{Salford, England}
      {BSc Physics in Physics}{September 2022 -- June 2025}
    \resumeSubheading
      {Ashton Sixth Form College}{Ashton-under-Lyne, England }
      {Maths, Physics, Computer Science}{September 2020 -- July 2022}
  \resumeSubHeadingListEnd
  
%-----------PROJECTS-----------
\section{Projects}
    \resumeSubHeadingListStart
          \resumeProjectHeading
          {\textbf{Producing and Engineering a Drone} $|$ \emph{C ++, Fusion 360, Arduino, Tinkercad}}{September 2023 -- June 2024}
          \resumeItemListStart
            \resumeItem{Arduino Programming: Developed and programmed the
drone’s flight control system using Arduino, implementing
real-time control algorithms using C++}
            \resumeItem{Circuit Design and Hardware Assembly: Designed and
assembled the drone’s electronic components, including the
motors, ESCs, and flight controller}
            \resumeItem{CAD modelling with Fusion 360: Designed and modelled custom drone parts using Fusion 360 to ensure precise fit and functionality }
            \resumeItem{Troubleshooting and Debugging: Diagnosed and resolved hardware issues with the Arduino chip and software issues in C++ during the
testing and development phase}
          \resumeItemListEnd
          \resumeProjectHeading
          {\textbf{Programming Quantum Computers} $|$ \emph{IBM Quantum, PennyLane, Microsoft Azure}}{October 2024 -- Present}
          \resumeItemListStart
            \resumeItem{Gained hands-on experience with a variety of quantum technologies, including superconducting qubits (IBM Quantum), photonic systems (Xanadu), trapped ions (IonQ), and quantum annealing (D-Wave)}
          \resumeItemListEnd
            \resumeProjectHeading
          {\textbf{Non-Analytic Basins and Chaotic System Simulation} $|$ \emph{MATLAB}}{September 2024 -- October 2024}
          \resumeItemListStart
            \resumeItem{Numerical Solution Using ODE Solver: Used MATLAB’s ode45 function to numerically solve the ODEs for both original and perturbed initial conditions over a defined time span}
            \resumeItem{Visualisation of Chaotic Dynamics: Generated time-series plots to illustrate the sensitivity to initial conditions and a 3D plot to visualise the chaotic attractor in phase space}
          \resumeItemListEnd
      \resumeProjectHeading
          {\textbf{Lennard-Jones Potential Plotting} $|$ \emph{Python, NumPy, Matplotlib}}{October 2023 -- November 2024}
          \resumeItemListStart
            \resumeItem{Data Processing and Analysis: Loaded and processed atomic separation and potential energy data from CSV files using Pandas for data manipulation}
            \resumeItem{Mathematical Modelling: Calculated the Lennard-Jones potential values using NumPy, implementing the potential energy formula to model interatomic forces}
            \resumeItem{Data Visualisation: Plotted experimental data and theoretical potential curve using Matplotlib, comparing atomic separation with potential energy}
            \resumeItem{Algorithm Optimisation: Utilised NumPy's linspace to generate smooth, high-resolution data points for accurate visualisation of potential curves}
          \resumeItemListEnd
      \resumeProjectHeading
          {\textbf{Simple Harmonic Motion Simulation} $|$ \emph{Python, VPython, GitHub}}{September 2023 -- May 2023}
          \resumeItemListStart
            \resumeItem{Applied Python libraries like VPython to create interactive visualisations and simulations,
demonstrating proficiency in using tools for scientific computing}
            \resumeItem{Created a physics simulation in VPython to model and analyse Simple Harmonic Motion (SHM)}
            \resumeItem{Used object-oriented programming to structure and optimize the simulation model}
            \resumeItem{Gained experience in testing and debugging code by analysing data-driven results to refine simulation parameters}
          \resumeItemListEnd
    \resumeSubHeadingListEnd

%-----------EXPERIENCE-----------
\section{Experience}
  \resumeSubHeadingListStart
% -----------Multiple Positions Heading-----------
%    \resumeSubSubheading
%     {Software Engineer I}{Oct 2014 - Sep 2016}
%     \resumeItemListStart
%        \resumeItem{Apache Beam}
%          {Apache Beam is a unified model for defining both batch and streaming data-parallel processing pipelines}
%     \resumeItemListEnd
%    \resumeSubHeadingListEnd
%-------------------------------------------

    \resumeSubheading
      {Information Technology Support Specialist}{September 2018 -- August 2019}
      {Microtech Computers}{Ashton-under-Lyne, England}
      \resumeItemListStart
        \resumeItem{Operating System Installation: Installed and configured
operating systems via BIOS, ensuring smooth setup and
performance}
        \resumeItem{System Testing: Performed post-repair tests to verify system stability and ensure that repairs were successful }
    \resumeItemListEnd

%


%-------------------------------------------
\end{document}
